\documentclass{article}
\usepackage{fullpage}
\usepackage{amsmath}
\usepackage{graphicx}
\usepackage[at]{easylist}
\headsep=15pt

\usepackage{datetime}
\mdyydate
\usepackage{fancyhdr}
\pagestyle{fancy}
\rhead{\today: \currenttime}
\renewcommand{\headrulewidth}{0pt}
\setlength{\headheight}{20pt}
\setlength{\headsep}{0.3in}

\begin{document}
\begin{center}{\huge\bf GTrends and the Market: Rough Notes}\end{center}

I'm particularly interested in extreme events in Google trends data. It
seems very possible that the only fairly pronounced spikes are really
going to clear the noise threshold.

Right now I'm only matching stock tickers with searches for the stock ticker.
Probably implementing company names would be at least as interesting.

Many stock tickers are also common words in various languages. I haven't restricted 
the google trends data to just the U.S., although I probably should since I'm using
U.S. stock tickers. I could eliminate companies like Allete Inc., a midwest energy company
with stock symbol ALE, since most searches for ale probably have nothing to do with the company,
but I'm leaving it in for now as a sanity check on my filters. 

Google limits you to a couple hundred google trends reports at a time the way I currently
have this implemented. That's not a huge problem at the moment, but it's definitely a
limitation to be aware of.

I'm not sure how quandl WIKI data handles stock splits. I may need to look 
out for that.

This is still really exploratory. I'm throwing out models

I may try and integrate/use some of the stuff that is part of the pytrends module.
\url{https://github.com/GeneralMills/pytrends}
It has some nice features, but I'm am a little uncomfortable with code, especially
code I didn't write, that includes my gmail username/password in plaintext. 
At the moment, I'm more comfortable just leaving the browser logged in.
The pyTrends module also results in google notifying me of each individual "log-in"
from unrecognized machines, which is sort of annoying.

Should read search popularity into a dictionary with the date as the key, search popularity as the value
The pandas data frame that contains the Quandl data has these date keys in the first column, and doing this allows
us to add the correct values for search frequency to the data frame 


\begin{easylist}
@ Potential variables
@@ Largest week on week change in trend interest
@@ Time series cross correlation
\end{easylist}

\end{document}
